\begin{Name}{1}{wmii}{Kris Maglione}{}{wmii - window manager improved, improved}
	\Prog{wmii}-VERSION
\end{Name}

\section{SYNOPSIS}
\Prog{wmii} \oOptArg{-a}{<address>} \oOptArg{-c}{<wmiirc>} \\
\Prog{wmii} \Opt{-v}

\section{DESCRIPTION}

\subsection{Overview}

\Prog{wmii} is a dynamic window manager for X11. In contrast to
static window management the user rarely has to think about how to
organize windows, no matter what he is doing or how many
applications are used at the same time.  The window manager adapts
to the current environment and fits to the needs of the user, rather
than forcing him to use a preset, fixed layout and trying to
shoehorn all windows and applications into it.

\Prog{wmii} supports classic and tiled window management with
extended keyboard and mouse control. The classic window management
arranges windows in a floating layer in which windows can be moved
and resized freely. The tiled window management is based on columns
which split up the screen horizontally. Each column handles
arbitrary windows and arranges them vertically in a non\-overlapping
way. They can then be moved and resized between and within columns
at will.

\Prog{wmii} provides a virtual filesystem which represents the
internal state similar to the procfs of Unix operating systems.
Modifying this virtual filesystem results in changing the state of
the window manager. The virtual filesystem service can be accessed
through 9P\-capable client programs, like \Cmd{wmiir}{1}.  This
allows simple and powerful remote control of the core window
manager.

\Prog{wmii} basically consists of clients, columns, views, and the
bar, which are described in detail in the \textbf{Terminology}
section.

\subsection{Terminology}

\begin{description}
\item[Display] A running X server instance consisting of input devices
	and screens.
\item[Screen] A physical or virtual (Xinerama or \Cmd{Xnest}{1}) screen
	of an X display. A screen displays a bar window and a view at a time.
\item[Window] A (rectangular) drawable X object which is displayed on a
	screen, usually an application window.
\item[Client] An application window surrounded by a frame window containing
	a border and a titlebar.
\item[Floating layer] A screen layer of \Prog{wmii} on top of all other
	layers, where clients are arranged in a classic (floating) way.
	They can be resized or moved freely.
\item[Managed layer] A screen layer of \Prog{wmii} behind the floating layer,
	where clients are arranged in a non\-overlapping (managed) way.  Here,
	the window manager dynamically assigns each client a size and position.
	The managed layer consists of columns.
\item[Tag] Alphanumeric strings which can be assigned to a client. This provides
	a mechanism to group clients with similar properties. Clients can have one
	tag, e.g. \emph{work}, or several tags, e.g.  \emph{work+mail}.

	Tags are separated with the \emph{+} character.
\item[View] A set of clients containing a specific tag, quite similiar to a
	workspace in other window managers.  It consists of the floating and
	managed layers.
\item[Column] A column is a screen area which arranges clients vertically in a
	non-overlapping way. Columns provide three different modes, which
	arrange clients with equal size, stacked, or maximized respectively.
	Clients can be moved and resized between and within columns freely.
\item[Bar] The bar at the bottom of the screen displays a label for each view and
	allows the creation of arbitrary user\-defined labels.
\item[Event] An event is a message which can be read from a special file in the
	filesystem of \Prog{wmii}, such as a mouse button press, a key press, or
	a message written by a different 9P-client.
\end{description}

\subsection{Basic window management}

Running a raw \Prog{wmii} process without a \Cmd{wmiirc}{1} script
provides basic window management capabilities already. However, to
use it effectively, remote control through its filesystem interface
is necessary.  By default it is only usable with the mouse in
conjunction with the \emph{Mod1 (Alt)} modifier key. Other
interactions, such as customizing the style, killing or retagging
clients, and grabbing keys, cannot be achieved without accessing the
filesystem.

The filesystem can be accessed by connecting to the \emph{address}
of \Prog{wmii} with any 9P-capable client, such as \Cmd{wmiir}{1}

\subsection{Actions}

An action is a shell script in the default setup, but it can
actually be any executable file.  It is executed usually by
selecting it from the actions menu.  You can customize an action by
copying it from the global action directory
\File{CONFPREFIX/wmii-3.5} to \File{\$HOME/.wmii-3.5} and then
editing the copy to fit your needs.  Of course you can also create
your own actions there; make sure that they are executable.

Here is a list of the default actions:

\begin{Table}[]{2}
quit	& leave the window manager nicely \\
status	& periodically print date and load average to the bar \\
welcome	& display a welcome message that contains the wmii tutorial \\
wmiirc	& configure wmii \\
\end{Table}

\subsection{Default Key Bindings}
\subsubsection{Moving Around}
\begin{Table}[]{2}
\textbf{Key} & \textbf{Action} \\
Mod-h	& Move to a window to the \emph{left} of the one currently
	focused \\
Mod-l	& Move to a window to the \emph{right} of the one currently
	focused \\
Mod-j	& Move to the window \emph{below} the one currently focused \\
Mod-k	& Move to a window \emph{above} the one currently focused \\
Mod-space & Toggle between the managed and floating layers \\
Mod-t \emph{tag} & Move to the view of the given \emph{tag} \\
Mod-\emph{[0-9]} & Move to the view with the given number \\
\end{Table}

\subsubsection{Moving Things Around}
\begin{Table}[]{2}
\textbf{Key} & \textbf{Action} \\
Mod-Shift-h	& Move the current window \emph{window} to a
	column on the \emph{left} \\
Mod-Shift-l	& Move the current window to a column
	on the \emph{right} \\
Mod-Shift-j	& Move the current window below the window
	beneath it. \\
Mod-Shift-k	& Move the current window above the window
	above it. \\
Mod-Shift-space	& Toggle the current window between the
	managed and floating layer \\
Mod-Shift-t \emph{tag}	& Move the current window to the
	view of the given \emph{tag} \\
Mod-Shift-\emph{[0-9]}	& Move to the current window to the
	view with the given number \\
\end{Table}

\subsubsection{Miscellaneous}
\begin{Table}[]{2}
\textbf{Key} & \textbf{Action} \\
Mod-m	& Switch the current column to \emph{max mode} \\
Mod-s	& Switch the current column to \emph{stack mode} \\
Mod-d	& Switch the current column to \emph{default mode} \\
Mod-Shift-c	& \Prog{Kill} the selected client \\
Mod-p \emph{program}	& \Prog{Execute} \emph{program} \\
Mod-a \emph{action}	& \Prog{Execute} the named \emph{action} \\
Mod-Enter	& \Prog{Execute} an \Prog{xterm} \\
\end{Table}

\section{Configuration}

If you feel the need to change the default configuration, then
customize (as described above) the \Prog{wmiirc} action.  This
action is executed at the end of the \Prog{wmii} script and does all
the work of setting up the window manager, the key bindings, the bar
labels, etc.

\section{FILES}

\begin{description}
\item[/tmp/ns.$USER.${DISPLAY\%.0}/wmii] The wmii socket file which provides a 9P service.
\item[CONFPREFIX/wmii-3.5] Global action directory.
\item[\$HOME/.wmii-3.5] User-specific action directory.  Actions are first searched here.
\end{description}

\section{ENVIRONMENT}

\begin{description}
\item[HOME, DISPLAY] See the section \textbf{FILES} above.
\end{description}

The following variables are set and exported within \Prog{wmii} and
thus can be used in actions:

\begin{description}
\item[WMII\_ADDRESS] Socket file of Used by \Cmd{wmiir}{1}.
\end{description}

\section{SEE ALSO}
\Cmd{dmenu}{1}, \Cmd{wmiir}{1}

