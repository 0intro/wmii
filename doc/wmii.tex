% (C)opyright 2005 by Anselm R. Garbe
\documentclass{article} \usepackage{times} \begin{document}

\title{Improved GUI concepts for experienced users}

\author{Anselm R. Garbe\\ \small garbeam at gmail dot com}

\maketitle \thispagestyle{empty}

\begin{abstract}
This article presents the motivation and concepts of the dynamic
window manager wmii and the graphical toolkit liblitz for the
\it{X Window System}.
\end{abstract}


%------------------------------------------------------------------------- 
\section{Motivation}

Most common graphical user interfaces are designed after the WIMP\cite{wimp}
paradigm, which has dominated the desktop environment
landscape since late 1980s. While research has been done on alternative
user interfaces, often the focus targeted more in ease of use and low
learning curves for new computer users rather than in efficiency and
power of abstraction.

The main reason has been the economical success of computers
in the normal consumer market, which consists of unexperienced users mainly.
Our motivation is to change this situation and to provide a graphical
user interface for experienced users, though we know that this market is
a niche.

There has been done rarly research in the non-wimp GUI landscape for years.
Back in 80s and early 90s there has been some research in
this area for the Plan 9\cite{plan9} operating system at Bell Labs. 
Most recent research has been done by individuals only, like Tuomo Valkonen with
his Ion\cite{ion} project and Lars Bernhardsson with his
LarsWM\cite{larswm} project.

The approaches found in the Plan 9 operating system are interesting, because 
on the one hand they cancelled the Unix tradition to work in Teletype emulators
and on the other hand, they didn't followed the WIMP paradigm propagated by Apple,
IBM or Microsoft. This makes Plan 9 the most unique approach compared to the
classical WIMP world.

The main aspects of an improved GUI consists of two things, a powerful
window management approach and a sane and simple widget set which fits well
into this window management approach.

In the area of improved window management concepts there has been done more
research, thus there appeared several different approaches. But the area
of improved widget sets which form powerful applications with a simple widget
set has been ignored for long time. Instead, the WIMP world introduced many
widgets which seem to focus on eye-candy, like progress bars, but don't
fix the essential problems with WIMP toolkits.



\section{Future}


\section*{Acknowlegdements} Following people provided useful feedback or several
grammar fixes to this article:
\begin{itemize}
\item Frank Boehme (1st version of this article)
\item Tuncer M zayamut Ayaz (1st version of this article)
\end{itemize}

\begin{thebibliography}{99}
\bibitem{wimp} Ashley George Taylor, WIMP Interfaces, CS6751 Topic Report: Winter '97
\bibitem{x} X Window System - http://www.freedesktop.org
\bibitem{plan9} plan9 operating system - http://cm.bell-labs.com/plan9dist/
\bibitem{acme} Rob Pike, Acme: A User Interface for Programmers -
http://www.cs.bell-labs.com/sys/doc/acme/acme.html
\bibitem{rat} Ratpoison window manager - http://www.nongnu.org/ratpoison/
\bibitem{evil} evilwm window manager - http://evilwm.sf.net
\bibitem{ion} Ion window manager - http://modeemi.cs.tut.fi/~tuomov/ion/
\bibitem{larswm} LarsWM window manager - http://www.fnurt.net/larswm/
\bibitem{vi} Vi Improved (VIM) - http://www.vim.org
\bibitem{9p} 9P protocol - http://www.cs.bell-labs.com/sys/man/5/INDEX.html
\end{thebibliography}

\end{document}
